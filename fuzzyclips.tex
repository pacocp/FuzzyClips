%%%%%%%%%%%%%%%%%%%%%%%%%%%%%%%%%%%%%%%%%
% Beamer Presentation
% LaTeX Template
% Version 1.0 (10/11/12)
%
% This template has been downloaded from:
% http://www.LaTeXTemplates.com
%
% License:
% CC BY-NC-SA 3.0 (http://creativecommons.org/licenses/by-nc-sa/3.0/)
%
%%%%%%%%%%%%%%%%%%%%%%%%%%%%%%%%%%%%%%%%%

%----------------------------------------------------------------------------------------
%	PACKAGES AND THEMES
%----------------------------------------------------------------------------------------

\documentclass{beamer}
\usepackage[spanish]{babel}
\usepackage[utf8]{inputenc}
\usepackage{hyperref}
\usepackage{fancyvrb}
\mode<presentation> {

% The Beamer class comes with a number of default slide themes
% which change the colors and layouts of slides. Below this is a list
% of all the themes, uncomment each in turn to see what they look like.

%\usetheme{default}
%\usetheme{AnnArbor}
%\usetheme{Antibes}
%\usetheme{Bergen}
%\usetheme{Berkeley}
%\usetheme{Berlin}
%\usetheme{Boadilla}
%\usetheme{CambridgeUS}
%\usetheme{Copenhagen}
%\usetheme{Darmstadt}
%\usetheme{Dresden}
%\usetheme{Frankfurt}
%\usetheme{Goettingen}
%\usetheme{Hannover}
%\usetheme{Ilmenau}
%\usetheme{JuanLesPins}
%\usetheme{Luebeck}
%\usetheme{Madrid}
%\usetheme{Malmoe}
%\usetheme{Marburg}
%\usetheme{Montpellier}
\usetheme{bjeldbak}
%\usetheme{Pittsbrgh}
%\usetheme{Rochester}
%\usetheme{Singapore}
%\usetheme{Szeged}
%\usetheme{Warsaw}

% As well as themes, the Beamer class has a number of color themes
% for any slide theme. Uncomment each of these in turn to see how it
% changes the colors of your current slide theme.

%\usecolortheme{albatross}
%\usecolortheme{beaver}
%\usecolortheme{beetle}
%\usecolortheme{crane}
%\usecolortheme{dolphin}
%\usecolortheme{dove}
%\usecolortheme{fly}
%\usecolortheme{lily}
%\usecolortheme{orchid}
%\usecolortheme{rose}
%\usecolortheme{seagull}
%\usecolortheme{seahorse}
%\usecolortheme{whale}
%\usecolortheme{wolverine}

%\setbeamertemplate{footline} % To remove the footer line in all slides uncomment this line
%\setbeamertemplate{footline}[page number] % To replace the footer line in all slides with a simple slide count uncomment this line

%\setbeamertemplate{navigation symbols}{} % To remove the navigation symbols from the bottom of all slides uncomment this line
}

\usepackage{graphicx} % Allows including images
\usepackage{booktabs} % Allows the use of \toprule, \midrule and \bottomrule in tables

%----------------------------------------------------------------------------------------
%	TITLE PAGE
%----------------------------------------------------------------------------------------

\title[FuzzyClips]{FuzzyClips} % The short title appears at the bottom of every slide, the full title is only on the title page

\author{Francisco Carrillo Pérez} % Your name
\institute[UGR] % Your institution as it will appear on the bottom of every slide, may be shorthand to save space
{
Universidad de Granada \\ % Your institution for the title page
\medskip

}
\date{\today} % Date, can be changed to a custom date

\begin{document}

\begin{frame}
\titlepage % Print the title page as the first slide
\end{frame}

\begin{frame}
\frametitle{Indice} % Table of contents slide, comment this block out to remove it
\tableofcontents % Throughout your presentation, if you choose to use \section{} and \subsection{} commands, these will automatically be printed on this slide as an overview of your presentation
\end{frame}

%----------------------------------------------------------------------------------------
%	PRESENTATION SLIDES
%----------------------------------------------------------------------------------------

\section{Introducción }
\begin{frame}
	\frametitle{Introducción}
	FuzzyClips es una extensión de CLIPS creada por el \textit{Integrated Reasoning Group of the Institute for Information Technology of the National Research Council of Canada }, que nos da la capacidad de representar y manipular reglas y hechos difusos.
\end{frame}

\section{Instalación}
\begin{frame}
	\frametitle{Instalación y Documentación}
	Mirror donde poder descargar los ejecutables de Windows y Linux:
	\url{http://awesom.eu/~quentin/archives/2010/04/22/fuzzyclips_downloads/index.html}.\\
	También se puede encontrar en el mismo enlace la documentación.
\end{frame}

\section{¿Cómo p*** se programa en esto?}

\subsection{Tranquilidad}
\begin{frame}
\frametitle{Tranquilidad}
La sintáxis es la misma que en CLIPS, solo se añaden ciertas funcionalidades que es lo que nos permite la representación de la incertidumbre.
\end{frame}

\subsection{Fuzzy Templates}

\begin{frame}[fragile]
	\frametitle{deftemplate}
	Todas las variables fuzzy deben declararse en un template antes de poder usarse. Se declara el template de la siguiente forma:
	\begin{Verbatim}[tabsize=8,fontsize=\footnotesize]
	(deftemplate <name> [“<comments>”]
	    <from> <to> [<unit>] ; rango de valores
	    (
	        t1
	        .
	        .    ; lista de los términos primarios
	        .    
         	tn
	    )
	)
	\end{Verbatim}

\end{frame}
\subsection{Fuzzy Facts}

\subsection{Fuzzy Rules}



%----------------------------------------------------------------------------------------

\end{document} 